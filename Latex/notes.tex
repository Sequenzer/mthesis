\documentclass[./main.tex]{subfiles}

\begin{document}

\section{Example class 1}
\begin{thm}
Let $p: \R \rightarrow \R$ with $p(x) = c e^{\alpha x}$ and $\alpha \in \R$ be a monomial. Then the following statement holds
\begin{align*}
c> 0 \implies  0 \leq f(x)+f(-x)-f_{min} \not\in SAGE(A,\R)
\end{align*}
für
\begin{align*}
f(x) = c_3 \cdot e^{(\alpha +1) x} -p(x) + c_1 \cdot e^{(\alpha -1) x} \geq 0 \ \forall x \in \mathbb{R}
\end{align*}
with $A = \myvec{\alpha +1 \\ \alpha \\ \alpha -1 }$, $c_1 = \frac {\alpha e} {2} $ and $c_3 = \frac {\alpha} {2 e}$. $f_{min}$ denotes the minimum of $f(x)+f(-x)$.
\begin{proof} First let $g=f(x)+f(-x)-f_{min}$. It is easily provable that  $f(1)=f'(1)=0$ which means every positive coefficient of monomials in $f$ is minimal in that, if we would decrease them $g$ would be negative. Furthermore there now is an extra negative coefficient which means $g\not \in SAGE(A,\R)$. 
\begin{align*}
c_g=\myvec{c_3\\-c\\c_2\\c_3\\-c\\c_2\\-f_{min}}
\end{align*}
Nonnegativity follows from nonegativity of $f$.
\end{proof}

\end{thm}
\begin{thm}
Let $p: \R^n \rightarrow \R$ with $p(x) = c e^{\alpha^T x}$ and $\alpha \in \R^n$ be a monomial. Then the following statement holds
\begin{align*}
c> 0 \implies \leq f(x)+f(-x)-f_{min} \not\in SAGE(A,\R^n)
\end{align*}
\begin{align*}
f(x) = \sum_i^n c_{3,i} \cdot e^{(\alpha_i +1) x_i} -p(x) + \sum_i^n c_{1,i} \cdot e^{(\alpha_i -1) x_i}
\end{align*}
with $A = \myvec{\alpha +\vec 1\\ \alpha \\ \alpha -\vec 1 }$, $c_{1,i} = \frac {\alpha_i e} {2} $ and $c_{3,i} = \frac {\alpha_i} {2 e}$.
\begin{proof}
Follows directly from above?
\end{proof}
\end{thm}
This can be generalized to $p$ being a general Posynomials... 


\section{Frage}
In \emph{Murray, Naumann, Theobald} werden ja die extreme Rays der SAGE Kegeln angesprochen, daher habe ich mir überlegt ob es eine Charakterisierung der Signome in  "Rand" des Kegels gibt. Vielleicht eine Überlegung in Richtung \emph{minimal SAGE}. Sowas in der Art
\begin{align*}
f(x) = \sum_{i=1}^n c_i exp(A_ix) \text{ ist SAGE}(A,\X)
\end{align*}
aber für alle $0<\epsilon \in \mathbb{R}^n$.(Hier $<0$ bzgl. minimaler Komponente)
\begin{align*}
f(x) = \sum_{i=1}^n (c_i- \epsilon_i) exp(A_ix) \text{ ist nicht SAGE}(A,\X)
\end{align*}
so dass man SAGE-Signome in eben diese minimalen Komponenten zerlegen könnte. 

\section{Sage-Positivstellensatz Example}
Using Wangs Positivstellensatz on his own Example.
\subsection{Signomial to Polynomial}
This is an Example from Wang et al. (Citation). Let $\mathcal X = \lbrace x \in \R^2 | x_1 = 1, -1\leq x_2 \leq 1 \rbrace$, 
\begin{equation*}
c = \c, \ A =  \A, \ f=e^{\frac 2 3 x_1} - e^{x_2} - e^{-x_2}
\end{equation*}
Translation $\X$ in a intersection of Halfspaces we get
\begin{align*}
\X = \lbrace x \in \R \vert Hx \leq d \rbrace
\end{align*}
with
\begin{align*}
H =  \H \text{ and } d = \d.
\end{align*}
To follow Wang's proof we first construct a leftinverse of $A$ and $B=HL$.
\begin{align*}
L = \L, B = \B
\end{align*}
Now it obviously holds that $Hx \leq d  \Leftrightarrow BAx \leq d$. If we transform $x$ by $y=\text{exp}(Ax)$ we get for every halfspace constraint a new polynomial constraint.
\begin{align*}
B_1(Ax) \leq d_1 \Leftrightarrow  \myvec{\frac 2 3\\0\\0}^T  \cdot Ax \leq d \Leftrightarrow y_1^{\frac 2 3}y_2^0y_3^0 \leq e^d
\end{align*}
Doing this we get a final semialgebraic set
\begin{align*}
\T = \lbrace y \in \R^2_+ \vert \ & e^3 - y_1^2 \geq 0 \\
& e^{-3}y_1^2 -1 \geq 0 \\
& e - y_2 \geq 0\\
& ey_2 - 1 \geq 0 \\
& y_1 - e^{\frac 3 2} \geq 0 \\
& e^{\frac 3 2} - y_1 \geq 0 \\
&e- y_2 \geq 0 \\
&y_2 -e^{-1}  \geq 0
\end{align*}
We use the fact that $y_3 = exp(-x_2) = y_2^{-1}.$ Homogenisation now leads to
\begin{align*}
\T = \lbrace y \in \R^3_+ \vert \ & y_3^2e^3 - y_1^2 \geq 0 \\
& e^{-3}y_1^2 -y_3^2 \geq 0 \\
& ey_3 - y_2 \geq 0\\
& ey_2 - y_3 \geq 0 \\
& y_1 - e^{\frac 3 2}y_3 \geq 0 \\
& e^{\frac 3 2}y_3 - y_1 \geq 0 \\
&ey_3- y_2 \geq 0 \\
&y_2 -e^{-1}y_3  \geq 0 \\
&\textcolor{red}{y_3 =  1}
\end{align*}
The last equation is impossible to homogenize but Dickinsons Positivstellensatz allows one polynomial to be constant. Now we only have to do the polynomials case....

\subsection{Dickinsons Positivstellensatz}
Pulling $\T$ through Dickinsons Positivstellensatz.\\
ToDo\\
$\vdots$



 
\end{document}
