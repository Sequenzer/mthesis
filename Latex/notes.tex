\documentclass[./main.tex]{subfiles}
\begin{document}
\section{Non-Sage Signomials}
Se $\mathbb{S}^1$ die Menge aller Signome . Wir können jedes Signom in die Differenz von 2 Posynome aufteilen.
\begin{equation}
f(x)= \sum_{j \in P } c_j x^{\alpha_j} + \sum_{j \in N} c_j x^{\alpha_h}
\end{equation}
mit $c_j<0$ wenn $j \in N$ und $c_j>0$ wenn $j \in P$. $\alpha_j \in \mathbb{R}$ sind reelle Zahlen.

\begin{lem}
$\left\vert P \right\vert < \left\vert N \right\vert$ implies $f \not\in SAGE$
\begin{proof}
Angenommen $f \in SAGE$. D.h es existieren $g_1,\dots,g_n$ mit $g_i \in AGE$.Sodass 
\begin{equation}
f(x)= \sum_{j =1}^n g_j(x)
\end{equation}
Insbesondere gilt $g_i>0$ und $g_i$ enthält höchstens ein negatives Monom. Also gilt $n \geq \left\vert N \right\vert > \left\vert P \right\vert$. Woraus folgt, dass mindestens ein $g_j$ kein positives Monom enthält und somit global negativ sein muss. Es folgt $g_j \not\in  AGE$. Ein Widerspruch.
\end{proof}
\end{lem}

\begin{lem}
Es gilt
\begin{equation}
f(x) = \left\vert c \right\vert x^m - \sum_{j\in P} c_j x^{\alpha_j} + \left\vert c \right\vert x^n >0 \ \forall x\in \mathbb{R_\geq}
\end{equation}
Wobei $ c=(c_1,\dots,c_n) \in \mathbb{R}_\geq^n$. Mit $m > max \lbrace \alpha_j \ \vert \  j \in P \rbrace$ und $n < min \lbrace \alpha_j \ \vert \  j \in P \rbrace$
\begin{proof}
Es gilt
\begin{equation}
\left\vert c \right\vert x^m \geq \sum_{j\in P} c_j x^{\alpha_j} \ \forall x \geq 1
\end{equation}
und 
\begin{equation}
\left\vert c \right\vert x^n > \sum_{j\in P} c_j x^{\alpha_j} \ \forall x < 1
\end{equation}
Folgt beides aus der Konstruktion. Somit sind wir fertig.
\end{proof}
\end{lem}
\begin{bsp}[$n=1$]
\begin{equation}
f(x) = 3 x^5 -x^\frac{25} 6  - x^\frac {17} 6 - x^\frac {11} 3   + 3x
\end{equation}
Ist nicht SAGE aber global positiv. Nach den Aussagen davor.
\end{bsp}
\begin{bsp}[$n=2$]
\begin{equation} 
f(x) = 3 x_1^5x_2^5 -x_1^\frac{25} 6 x_2^\frac {17} 6  - x_2^\frac {17} 6 x_2^\frac {11} 3 - x_3^\frac {11} 3 x_2^\frac{25} 6   + 3x_1x_2
\end{equation}
\color{red} Sollte im Allgemeinen Fall auch klappen. Habe es aber noch nicht so genau überpüft.....
\end{bsp}

\end{document}