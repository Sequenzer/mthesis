\documentclass[./main.tex]{subfiles}

\begin{document}
\section{Appendix}
Temporary Code dump....
\subsection{Code}
Construction of Linear Program
\begin{lstlisting}
using DynamicPolynomials,LinearAlgebra
using JuMP
import HiGHS

function getLP(poly,arrayOfPolynomials,var)
    d = maxdegree(poly)
    base=monomials(var,d)
    b=coefficients(poly,base)
    c=[]
    nVars=[]
    for f in arrayOfPolynomials
        deg=d - maxdegree(f)
        push!(nVars,size(monomials(var,deg))[1])
        l = []
        for g in monomials(var,deg)
            basepoly=g*f
            push!(l,coefficients(basepoly,monomials(var,d)))
        end
        l=transpose(reduce(vcat,transpose.(l)))
        push!(c,l)
    end
    c=transpose(reduce(vcat,transpose.(c)))
    
    #Create Linear Program
    m = size(b,1) #Number of Monomials in Polynom to Match
    LP = Model(HiGHS.Optimizer) # Initialize Model
    set_optimizer_attribute(LP, "log_to_console", "false") # disable debug info
    n=size(c,2) # Number of Variables
    @variable(LP, u[1:n]>=0)
    for i in 1:m
        @constraint(LP, sum(dot(u,c[i,:])) == b[i])
    end
    # print(LP)
    return (LP,u,b,c,nVars)
end
\end{lstlisting}
Solving Linear Program

\begin{lstlisting}
function getSolution(poly,arrayOfPolynomials,var)
    LP,u,b,c,n = getLP(poly,arrayOfPolynomials,var)
    # print(LP)
    JuMP.optimize!(LP)
    if JuMP.has_values(LP)
        sol=JuMP.value.(u)
        walk=1
        c=[]
        solPoly=[]
        for (j,i) in enumerate(n)
            coeff = view(sol,walk:walk+i-1)
            base = monomials(var,maxdegree(poly)-maxdegree(arrayOfPolynomials[j]))
            push!(solPoly,polynomial(coeff,base))
            push!(c,coeff)
            walk+=i
        end
        return (solPoly,c)
    else
        return false
    end
end
\end{lstlisting}
Procedure that runs through the degrees

\begin{lstlisting}
function Run(poly,arrayOfPolynomials,maxeval)
    var = variables([arrayOfPolynomials...,poly])
    for i in 1:maxeval
        e=polynomial(monomials(var,1))^i*poly
        Sol,c = getSolution(e,arrayOfPolynomials,var)
        if Sol != false
            return Sol,c
        end
    end    
    return false
end
\end{lstlisting}

\subsubsection{Signomial to Polynomial}

Construct support for the Polynomials constraints of $\mathcal{X}$.
\begin{lstlisting}
function getSupport(B,d)
    Supp=[]
    nvars=size(B,2)
    # print(nvars)
    for (i,row) in enumerate(eachrow(B))
        P=[]
        N=[]
        denom=lcm(denominator.(row))
        for (j,ele) in enumerate(row)
            if ele >= 0
                push!(P,(Int(ele*denom),j))
            else
                push!(N,(Int(-ele*denom),j))
            end
        end
        pdegree=sum(Int[x[1] for x in P])
        ndegree=sum(Int[x[1] for x in N])
        if pdegree>ndegree
            push!(N,(pdegree-ndegree,nvars+1))
        elseif pdegree<ndegree
            push!(P,(ndegree-pdegree,nvars+1))   
        end
        push!(Supp,(P,N,denom*d[i]))
    end
    return (Supp,nvars)
end
\end{lstlisting}


Get redundant constraints needed for positivity on $\mathcal{X}$
\begin{lstlisting}
function getRedCons(A,H,d,var)
    R=[]
    # Start with upper bounds
    C = findMinOnSet(A,H,d)
    nvars = size(C,1) #Number of  Variables
    for i in 1:nvars
        fun = var[i]-e^C[i][1]*var[nvars+1]
        push!(R,fun)
    end
    
    # Add lower bounds
     D = findMaxOnSet(A,H,d)
    for i in 1:size(D,1)
        fun = e^D[i][1]*var[nvars+1]-var[i]
        push!(R,fun)
    end
    return (R,nvars)
end
\end{lstlisting}

finalising and homogenization
\begin{lstlisting}
function getallConstraints(A,H,c,d)
    SAS,var = SASbyMatrix(A,H,c,d)
    RC,nvars = getRedCons(A,H,d,var)
    Sol = vcat(SAS, RC)
    # Add Homogenization Constraint
    push!(Sol,var[nvars+1])
    # Objective Function
    f= sum(c.*var[1:nvars])
    return (f,Vector(Sol))
    
end
\end{lstlisting}


\end{document}