\documentclass[./main.tex]{subfiles}

\begin{document}
\section{Positivstellesatz for homogenous polynomials}

Dickinson et all provides a certificate for polynomials that are positive on a semialgebraic Set.
\begin{thm}[Positivestellsatz Dickinson]
\begin{equation}
\forall \bar{f} \in \mathbb{R}[x] \left( A(\bar{f}) \wedge B(\bar{f}) \right) \Rightarrow \exists r, \bar{g} \  C(r,\bar{g})
\end{equation}
where $A$ denotes a sentence to enforce $f_i$ do be homogenous
\begin{align*}
A(\bar{f}) \equiv \bar f \text{ is homogenous}.
\end{align*}
Sentence $B$ asks is our positivity Statement
\begin{align*}
B(\bar f ) \equiv \forall x \in \mathbb{R}_+\cap \bigcap_{i=1}^n f^{-1}(\mathbb R_+) \setminus \lbrace 0 \rbrace
\end{align*}
and finally C is the certificate for said positivity
\begin{align*}
C(\bar f ,r, \bar g) \equiv \forall x \in \mathbb{R}^m : \ \vert x \vert_1^r \cdot f_0(x) = \sum_i^n f_i(x)g_i(x) 
\end{align*}
with $\bar g$ having only positive coefficients.
\begin{proof}
ToDo
\end{proof}
\end{thm}
This means if we have a bound on the degree $r$ we calculate this represtation via a linear programm.

\begin{algorithm}
\caption{Brute force approach for a certificate}\label{euclid}
\begin{algorithmic}[1]
\SetKwProg{try}{try}{:}{}
\SetKwProg{catch}{catch}{:}{}
\State $\textit{LP} \gets \text{Linear Programm for A}$\\
\While {$i \leq r$} {
 \try{}{
	\Return Solution of \textit{LP}
 }
 \catch{LP infeasibel}{
 	$i++$
 }
}
\end{algorithmic}
\end{algorithm}


It is possible to construct a Linear Program that constructs the polynomials in the certificate C.
\subsection{Linear programm for the solution of Dickinson}
Let $\bar f = (f_0,\dots,f_n)$ be homogenous polynomials of degree $deg(f_i) = d_i$ that fulfill conditions $A$ and $C$. We construct polynomials $\bar g$ via equating of coefficients in $C$. To start we see that $\vert x \vert_1^r \cdot f_0(x)$ is a homogenous polynomial of degree $d_0+r$.
\begin{align*}
\vert x \vert_1^r \cdot f_0(x) = \sum_{\vert \alpha \vert = d_0+r} l_\alpha x^\alpha
\end{align*}
which means it is constructed via less than $m^{d_0+r}$ many. To match the degree on the left Side. $g_i$ has to have a degree of $d_0+r-d_i$, which means it is based on $m^{d_0+r+d_i}$ many monomials.

Now let $h_{(i,j)}$ be a monomial of degree $d_0+r-d_i$ and $f_i$ a homogenous polynomials as given. It holds that
\begin{align*}
h_{i,j} \cdot f_i = \sum_{\vert \alpha \vert = d_i} c_{i,j} b_{i,\alpha} x^\alpha
\end{align*}
By arbitrarily ordering every Monomials we can relabel $b_\alpha$ with $b_{j}$ refering to the $j$ position in our Order. 
Which means 
\begin{align*}
g_i \cdot f_i &= \sum_{j=1}^{m^{d_0+r-d_i}}  \sum_{\vert \alpha \vert = d_i} c_{i,j} b_{i,\alpha} x^{\alpha+\beta_j}\\
\Rightarrow \ \sum_i g_i \cdot f_i &=  \sum_{i=1}^n \sum_{j=1}^{m^{d_0+r-d_i}}  \sum_{\vert \alpha \vert = d_i} c_{i,j} b_{i,\alpha} x^{\alpha+\beta_j}
\end{align*}
has to hold. THis means to get the coefficients $l_\alpha$, we have to solve the following equation
\begin{align*}
l_\gamma &=  \sum_{i=1}^n \sum_{j=1}^{m^{d_0+r-d_i}}  \sum_{ \alpha: \alpha + \beta_j = d_i} c_{i,j} b_{i,\alpha} x^{\alpha+\beta_j}
\end{align*}
\end{document}

