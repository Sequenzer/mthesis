\documentclass[./main.tex]{subfiles}

\begin{document}
\section{Positivstellesatz for homogenous polynomials}

Dickinson et all provides a certificate for polynomials that are positive on a semialgebraic Set.
\begin{thm}[Positivestellsatz Dickinson]
\label{Dickinson}
\begin{equation}
\forall \bar{f} \in \mathbb{R}[x] \left( A(\bar{f}) \wedge B(\bar{f}) \right) \Rightarrow \exists r, \bar{g} \  C(r,\bar{g})
\end{equation}
where $A$ denotes a sentence to enforce $f_i$ do be homogenous
\begin{align*}
A(\bar{f}) \equiv \bar f \backslash f_1 \text{ is homogenous}.
\end{align*}
Sentence $B$ asks is our positivity Statement
\begin{align*}
B(\bar f ) \equiv \forall x \in \mathbb{R}_+\cap \bigcap_{i=1}^n f_i^{-1}(\mathbb R_+) \setminus \lbrace 0 \rbrace \Rightarrow f_0(x) >0 
\end{align*}
and finally C is the certificate for said positivity
\begin{align*}
C(\bar f ,r, \bar g) \equiv \forall x \in \mathbb{R}^m : \ \vert x \vert_1^r \cdot f_0(x) = \sum_i^n f_i(x)g_i(x) 
\end{align*}
with $\bar g$ having only positive coefficients.
\begin{proof}
ToDo
\end{proof}
\end{thm}
This means if we have a bound on the degree $r$ we calculate this represtation via a linear programm.

\begin{algorithm}
\caption{Brute force approach for a certificate}\label{euclid}
\begin{algorithmic}[1]
\SetKwProg{try}{try}{:}{}
\SetKwProg{catch}{catch}{:}{}
\State $\textit{LP} \gets \text{Linear Programm for A}$\\
\While {$i \leq r$} {
 \try{}{
	\Return Solution of \textit{LP}
 }
 \catch{LP infeasibel}{
 	$i++$
 }
}
\end{algorithmic}
\end{algorithm}


It is possible to construct a Linear Program that constructs the polynomials in the certificate C.
\subsection{Linear programm for the solution of Dickinson}


Let $\bar f = (f_0,\dots,f_n)$ be homogenous polynomials of degree $deg(f_i) = d_i$  that fulfill conditions $A$ and $C$. We construct polynomials $\bar g$ via equating of coefficients in $C$. To start we see that $\vert x \vert_1^r \cdot f_0(x)$ is a homogenous polynomial of degree $d_0+r$.
\begin{align*}
\vert x \vert_1^r \cdot f_0(x) = \sum_{\vert \alpha \vert = d_0+r} l_\alpha x^\alpha
\end{align*}
which means it is constructed via less than $\binom{m+ d_0+r -1 }{d_0+r}$ many monomials. To match the degree on the left side. $g_i$ has to have a degree of $d_0+r-d_i$, which means it is based on $ \binom {m + d_0+r-d_i-1} {d_0+r-d_i}$ many monomials.

Now let $h_{i,\beta}$ be a monomial of degree $d_0+r-d_i$ and $f_i$ a homogenous polynomials as given. It holds that
\begin{align*}
h_{i,\beta} \cdot f_i = \sum_{\vert \alpha \vert = d_i} c_{i,\beta} b_{i,\alpha} x^\alpha
\end{align*}
Let $g_i$ be the homogenous Polynomial of degree $d_0+r-d_i$. This leads to
\begin{align*}
g_i \cdot f_i &= \sum_{\vert \beta \vert = d_0+r-d_i}  \sum_{\vert \alpha \vert = d_i} c_{i,\beta} \cdot b_{i,\alpha} \cdot x^{\alpha+\beta}\\
\Rightarrow \ \sum_i g_i \cdot f_i &=  \sum_{i=1}^n \sum_{\vert \beta \vert = d_0+r-d_i}  \sum_{\vert \alpha \vert = d_i} c_{i,\beta} \cdot b_{i,\alpha} \cdot x^{\alpha+\beta}.
\end{align*}
To get a sense of scale for our linear programm we call recall that $b_{i,\alpha} = 0 $ is possible which means the worst case happens when  $b_{i,\alpha} \neq 0 $ for all $\alpha$ with $\vert \alpha \vert = d_0+r -d_i$. As we have already seen the number of $\alpha\in \mathbb{N}^m$ who fulfill this constraint are $ \binom {m +d_0+r -d_i -1 } {d_0+r -d_i}$,  which implies that $d_i$ has to be minimal in the worst case.

\subsection{Constructing a linear program}
With aboves reasoning we can construct a linear program of the form
\begin{equation*}
\begin{array}{ll@{}ll}
\text{feasibility}  & &c &\\
\text{subject to}& \sum_i B_i &c_i = l,  &\\
& &c_i \geq 0,  &\\
                 &&c_i \in \mathbb{R}^{d_0+r-d_i}.              
\end{array}
\end{equation*}
with $l = (l_{\gamma_1},\dots,l_{.})^T$ being the coefficient vector of $\vert x \vert_1^r \cdot f_0(x)$ as above. And $B_i$ is constructed as followed
\begin{equation*}
B_i[\gamma,\alpha]    = \begin{cases}
    b_{i,\alpha},& \text{if } \exists \beta \in \mathbb{N}^{d_0+r-d_i} : \alpha + \beta = \gamma\\
    0,              & \text{otherwise}
\end{cases}
\end{equation*}
with $\alpha \in \mathbb{N}^d_i$. 

Solving this lienar program gives us $n$ nonnegative coefficient vectors that describes the nonegative polynommials $g_i$ from theorem \ref{Dickinson}. 

\subsection{Example}
Let $\bar{f}=(f_0,\dots,f_4)$ the vector comprised of 
\begin{align*}
f_0 &= 3x_1 -2x_2 -2x_3 \\
f_1 &= 1\\
f_2 &= x_1 -x_2\\
f_3 &= x_1 -x_3\\
f_4 &= x_1^2 -4x_2x_3
\end{align*}
$\bar{f}$ is homogenous and $B(\bar{f})$ holds as well. Therefore $C(\bar{f},r,\bar{g})$ has to hold for some $r\in \mathbb{N}$ and homogenous polynomials $\bar{g}$ with nonnegative coefficients. 

Folowing aboves strategy we select $r=1$ and calculate $ \vert x \vert_1^1 \cdot f_0(x)$.
\begin{align}
\label{Example Dick}
 \vert x \vert_1^1 \cdot f_0(x) =  3x_1^2 +x_1x_2 +x_1x_3 -2x_2^2 - 4x_2x_3 - 2x_3^2
\end{align}
This already shows us that we have $\binom{3}{2}$ constraints. Now we have to calculate each $B_i$ in the linear program. Starting with $B_1$ we get that $B_1 \in \mathbb{R}^{6 \times 6}$. With the degree of $f_0$ beeing zero there is obviously only one $\alpha$ and in particular only one $b_{1,\alpha}= 1$. This means only one $\beta = \gamma$ exists with $\alpha + \beta = \gamma$ and $B_1$ is of the form
\begin{align*}
B_1 = \begin{pmatrix}
1&0&0&0&0&0\\
0&1&0&0&0&0\\
0&0&1&0&0&0\\
0&0&0&1&0&0\\
0&0&0&0&1&0\\
0&0&0&0&0&1\\
\end{pmatrix}.
\end{align*}
Calculating $B_2$ isn't much different. At first we observe that $g_2$ has only $\binom 3 1 = 3 $ (This is wrong!!). monomials which leads to $B_2 \in \mathbb{R}^6 \times 2$. $f_2 = x_1 -x_2$ which means the monomials
\begin{align*}
x_1,x_2,x_3,-x_1,-x_2,-x_3
\end{align*}
lead to monomials in equation \ref{Example Dick}. Following this 
\begin{align*}
B_2 = \begin{pmatrix}
1&0&0\\
-1&1&0\\
0&0&1\\
0&-1&0\\
0&0&-1\\
0&0&0\\
\end{pmatrix}.
\end{align*}
has 4 non zero elements. $B_3$ works the same way.
\begin{align*}
B_3 = \begin{pmatrix}
1&0&0\\
0&1&0\\
-1&0&1\\
0&0&0\\
0&-1&\\
0&0&-1\\
\end{pmatrix}.
\end{align*}
With $f_4$ having the same degree $d_4=d_0+r$ as needed, there is only the constant monnomial to construct $B_4$
\begin{align*}
B_4 = \begin{pmatrix}
1\\
0\\
0\\
0\\
-4\\
\end{pmatrix}.
\end{align*}

If we now solve the program
\begin{equation*}
\begin{array}{ll@{}ll}
\text{feasibility}  & &c &\\
\text{subject to}& \sum_i B_i &c_i = l,  &\\
& &c_i \geq 0,  &\\
                 &&c_i \in \mathbb{R}^{d_0+r-d_i}.              
\end{array}
\end{equation*}
we get
\begin{align*}
c_1 = \begin{pmatrix}
0\\0\\0\\0\\0\\0
\end{pmatrix},
c_2= \begin{pmatrix}
1 \\ 2 \\ 0
\end{pmatrix},
c_3= \begin{pmatrix}
1 \\ 0 \\ 2
\end{pmatrix}
c_4= \begin{pmatrix}
0
\end{pmatrix}.
\end{align*}
And the witness polynomials in Theorem \ref{Dickinson} are
\begin{align*}
\vert x \vert_1^1 \cdot f_0(x) = 0\cdot f_1 + (x_1+2x_2)\cdot f_2 +(x_1 + 2x_3)\cdot f_3 + 1 \cdot f_4
\end{align*}
\subsubsection{alogrithm}
ToDo


\end{document}

