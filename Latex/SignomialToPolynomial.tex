\documentclass[./main.tex]{subfiles}

\begin{document}
\section{SAGE Signomials}
\begin{ddef}[Signomial]
Let $c \in \mathbb{R}^l$ and $A \in \mathbb{l\times n}$. We then call the folowing sum of exponentiaal functions a \emph{Signomial}.
\begin{align*}
Sig(c,A)(x) = \sum_{j=1}^l c_j exp(A_jx)
\end{align*}
\end{ddef}
By using the transformation $y_i=exp(x_i)$ we can see the connection between Signomials and Polynomials. By interpreting the former as a generalized version of the latter. By expanding its exponents to real numbers.
\begin{align*}
Sig(c,A)(x) = \sum_{j=1}^l c_j exp(A_jx) = \sum_{j=1}^l c_j y_1^{A_{j,1}} \dots y_n^{A_{j,n}}
\end{align*}
\begin{ddef}[AGE Signomials]
We call a signomial with at most one negative coefficient an AGE-signomial over $\mathcal{X}$  which 
\begin{align*}
AGE(A,\mathcal{X}, i) = \lbrace Sig(c,A) | \ c_{\backslash i} \geq 0 \text{ and } Sig(c,a)(x) \geq 0 \ \forall x \in \mathcal{X} \rbrace
\end{align*}
\end{ddef}
\begin{ddef}[SAGE Signomials]
\emph{SAGE-signmials} are sums of AGE-signomials
\begin{align*}
SAGE(A,\mathcal{X}) = \sum_{i=1}^l AGE(A,\mathcal{X},i)
\end{align*}

\end{ddef}
\section{Positivstellensatz for SAGE Signomials}

\begin{thm} Let $c \in \mathbb{R}^n$ and $A = \left[A_1,\dots, A_l \right]^T \subset \mathbb{Q}^{l\times n}$. Additionaly let $\mathcal{X} \subset \mathbb{R}^n$ be a polyhedra on which $Sig(c,A)(x) > 0 \ \forall x \in \mathcal{X}$ holds. Then there exists some $p \in \mathbb{Z}_+$ such that $|exp(Ax)^T|^p_1 \cdot Sig(c,A)$ is a SAGE signomial.
\begin{proof}
ToDo
\end{proof}
\end{thm}
The proof of this is purely constructive and a bound on $p$ can easily computed given a upper bound for the degree in Dickinsons Positivstellensatz (Theorem \ref{Dickinson})

\section{Sage-Positivstellensatz Example}
Using Wangs Positivstellensatz on his own Example.
\subsection{Signomial to Polynomial}
This is an Example from Wang et al. (Citation). Let $\mathcal X = \lbrace x \in \R^2 | x_1 = 1, -1\leq x_2 \leq 1 \rbrace$, 
\begin{equation*}
c = \c, \ A =  \A, \ f=e^{\frac 2 3 x_1} - e^{x_2} - e^{-x_2}
\end{equation*}
Translation $\X$ in a intersection of Halfspaces we get
\begin{align*}
\X = \lbrace x \in \R \vert Hx \leq d \rbrace
\end{align*}
with
\begin{align*}
H =  \H \text{ and } d = \d.
\end{align*}
To follow Wang's proof we first construct a leftinverse of $A$ and $B=HL$.
\begin{align*}
L = \L, B = \B
\end{align*}
Now it obviously holds that $Hx \leq d  \Leftrightarrow BAx \leq d$. If we transform $x$ by $y=\text{exp}(Ax)$ we get for every halfspace constraint a new polynomial constraint.
\begin{align*}
B_1(Ax) \leq d_1 \Leftrightarrow  \myvec{\frac 2 3\\0\\0}^T  \cdot Ax \leq d \Leftrightarrow y_1^{\frac 2 3}y_2^0y_3^0 \leq e^d
\end{align*}
Doing this we get a final semialgebraic set
\begin{align*}
\T = \lbrace y \in \R^2_+ \vert \ & e^3 - y_1^2 \geq 0 \\
& e^{-3}y_1^2 -1 \geq 0 \\
& e - y_2 \geq 0\\
& ey_2 - 1 \geq 0 \\
& y_1 - e^{\frac 3 2} \geq 0 \\
& e^{\frac 3 2} - y_1 \geq 0 \\
&e- y_2 \geq 0 \\
&y_2 -e^{-1}  \geq 0
\end{align*}
We use the fact that $y_3 = exp(-x_2) = y_2^{-1}.$ Homogenisation now leads to
\begin{align*}
\T = \lbrace y \in \R^3_+ \vert \ & y_3^2e^3 - y_1^2 \geq 0 \\
& e^{-3}y_1^2 -y_3^2 \geq 0 \\
& ey_3 - y_2 \geq 0\\
& ey_2 - y_3 \geq 0 \\
& y_1 - e^{\frac 3 2}y_3 \geq 0 \\
& e^{\frac 3 2}y_3 - y_1 \geq 0 \\
&ey_3- y_2 \geq 0 \\
&y_2 -e^{-1}y_3  \geq 0 \\
&\textcolor{red}{y_3 =  1}
\end{align*}
The last equation is impossible to homogenize but Dickinsons Positivstellensatz allows one polynomial to be constant. Now we only have to do the polynomials case....

\end{document}