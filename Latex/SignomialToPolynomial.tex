\documentclass[./main.tex]{subfiles}

\begin{document}
\section{SAGE Signomials}
\begin{ddef}[Signomial]
Let $c \in \mathbb{R}^l$ and $A \in \mathbb{R}^{l \times n}$. We then call the following sum of exponentiaal functions a \emph{Signomial}.
\begin{align*}
Sig(c,A)(x) = \sum_{j=1}^l c_j exp(A_jx)
\end{align*}
\end{ddef}
By using the transformation $y_i=exp(x_i)$ we can see the connection between Signomials and Polynomials, by interpreting the former as a generalized version of the latter through expanding its exponents to real numbers.
\begin{align*}
Sig(c,A)(x) = \sum_{j=1}^l c_j exp(A_jx) = \sum_{j=1}^l c_j y_1^{A_{j,1}} \dots y_n^{A_{j,n}}
\end{align*}
\begin{ddef}[AGE Signomials]
We call a signomial with at most one negative coefficient an \emph{$\mathcal{X}$-SAGE} signomial,  with
\begin{align*}
AGE(A,\mathcal{X}, i) = \lbrace Sig(c,A) | \ c_{\backslash i} \geq 0 \text{ and } Sig(c,a)(x) \geq 0 \ \forall x \in \mathcal{X} \rbrace
\end{align*}
being the set of all \emph{$\mathcal{X}$-AGE} signomials with at most $c_i$ beeing (not necessarily) negative.
\end{ddef}
\begin{ddef}[SAGE Signomials]
\emph{$\mathcal{X}$-SAGE} signomials are sums of AGE-signomials
\begin{align*}
SAGE(A,\mathcal{X}) = \sum_{i=1}^l AGE(A,\mathcal{X},i)
\end{align*}

\end{ddef}
\section{Positivstellensatz for SAGE Signomials}
Similar to the polynomial case there exist Positivstellensätze for Signomials. The Statement here is a specific case of the SAGE-Positivstellensatz in \cite{wang2020positivstellensatz} from Wang et al. Our version differs by restricting its area of positivity to be a Polyhedral. This is necessary to avoid constructivistic issues with the Axiom of Choice that arise  via the Hahn–Banach theorem. Even though there are constructive version of the latter we will refrain from using this theorem altogether .
\begin{thm}[SAGE-Positivstellensatz]
\label{SAGEthm}

 Let $c \in \mathbb{R}^n$ and $A = \left[A_1,\dots, A_l \right]^T \subset \mathbb{Q}^{l\times n}$. Additionaly let $\mathcal{X} \subset \mathbb{R}^n$ be a polyhedra on which $Sig(c,A)(x) > 0 \ \forall x \in \mathcal{X}$ holds. Then there exists some $p \in \mathbb{Z}_+$ such that $|exp(Ax)^T|^p_1 \cdot Sig(c,A)$ is a SAGE signomial over $\mathcal{X}$.
\begin{proof}
Copy from Notes...
\end{proof}
\end{thm}
The part of the proof that doesn't rely on Dickinsons Positivstellensatz  is purely constructive, which means given a bound on $r$ in theorem \ref{Dickinson} we can easily compute a bound on $p$ in the SAGE-Positivstellensatz. This stems from the fact that similary to the polynomial case the modulations of $Sig(c,A)$ form a hierarchy i.e 
\begin{align*}
 |exp(Ax)^T|^p_1 \cdot Sig(c,A) \text{ is } \mathcal{X}\text{-SAGE } \Rightarrow |exp(Ax)^T|^{p+1}_1 \cdot Sig(c,A) \text{ is } \mathcal{X}\text{-SAGE }
\end{align*}
\section{Sage-Positivstellensatz Example}
Quick Example to explain the procedure.
\subsection{Signomial to Polynomial}
This is an Example from Wang et al. (Citation). Let $\mathcal X = \lbrace x \in \R^2 | x_1 = 1, -1\leq x_2 \leq 1 \rbrace$, 
\begin{equation*}
c = \c, \ A =  \A, \ f=e^{\frac 2 3 x_1} - e^{x_2} - e^{-x_2}
\end{equation*}
Translating $\X$ into an intersection of Halfspaces we get
\begin{align*}
\X = \lbrace x \in \R \vert Hx \leq d \rbrace
\end{align*}
with
\begin{align*}
H =  \H \text{ and } d = \d.
\end{align*}
To follow the proof we first construct a leftinverse of $A$ and $B=HL$.
\begin{align*}
L = \L, B = \B
\end{align*}
Now it obviously holds that $Hx \leq d  \Leftrightarrow BAx \leq d$. If we transform $x$ by $y=\text{exp}(Ax)$ we get for every halfspace constraint a new polynomial constraint.
\begin{align*}
B_1(Ax) \leq d_1 \Leftrightarrow  \myvec{\frac 2 3\\0\\0}^T  \cdot Ax \leq d \Leftrightarrow y_1^{\frac 2 3}y_2^0y_3^0 \leq e^d
\end{align*}
Doing this we get a final semialgebraic set
\begin{align*}
\T = \lbrace y \in \R^3_+ \vert \ & e^3 - y_1^2 \geq 0 \\
& e^{-3}y_1^2 -1 \geq 0 \\
& e - y_2 \geq 0\\
& ey_2 - 1 \geq 0 \\
& y_1 - e^{\frac 3 2} \geq 0 \\
& e^{\frac 3 2} - y_1 \geq 0 \\
&e- y_2 \geq 0 \\
&y_2 -e^{-1}  \geq 0 \\
&e- y_3 \geq 0 \\
&y_3 -e^{-1}  \geq 0 \rbrace.
\end{align*}
Homogenisation now leads to
\begin{align*}
\T' = \lbrace y \in \R^4_+ \vert \ & y_4^2e^3 - y_1^2 \geq 0 \\
& e^{-3}y_1^2 -y_4^2 \geq 0 \\
& ey_4 - y_2 \geq 0\\
& ey_2 - y_4 \geq 0 \\
& y_1 - e^{\frac 3 2}y_4 \geq 0 \\
& e^{\frac 3 2}y_4 - y_1 \geq 0 \\
&ey_4- y_2 \geq 0 \\
&y_2 -e^{-1}y_4  \geq 0 \\
&ey_4- y_3 \geq 0 \\
&y_3 -e^{-1}y_4  \geq 0 \\
&\textcolor{red}{y_4 =  1} \rbrace
\end{align*}
\subsection{Polynomial Solution}
The algorithm from Section \ref{secDick} terminates for $deg(r)=1$ and serves a solution
\begin{align*}
f=e^3 \cdot f_2 +0.068y_2  \cdot f_4 + (0.6y_2 +0.6y_3) f_5 + e^{\frac{3}{2}} y_4 \cdot f_6 + (1.11 y_1 +y_2) f_7 +(0.53 y_1 +y_3) \cdot f_{10}.
\end{align*}
Here $f_i$ denotes the respective polynomials in $\mathcal{T}'$.







\subsection{Example of positive but not $\mathcal{X}$-SAGE}
Theorem \ref{SAGEthm} is only useful if not every positive signomial (on $\mathcal{X}$) is immediatly $\mathcal{X}$-SAGE. Constructing examples of such signomials isn't necessarily trivial. At first we can construct a class of nonnegative but not $\mathcal{X}$-SAGE signomials.
\begin{thm}
\label{monomClass}
Let $p: \R \rightarrow \R$ with $p(x) = c e^{\alpha x}$ be a monomial with $\alpha >1$ and $c>0$.Then the following statement holds
\begin{align*}
0 \leq f(x)+f(-x)-f_{min} \not\in SAGE(A,\R)
\end{align*}
with
\begin{align*}
f(x) = c_3 \cdot e^{(\alpha +1) x} -p(x) + c_1 \cdot e^{(\alpha -1) x} \geq 0 \ \forall x \in \mathbb{R}
\end{align*}
with $A = \myvec{\alpha +1 \\ \alpha \\ \alpha -1 }$, $c_1 = \frac {c e} {2} $ and $c_3 = \frac {c} {2 e}$. $f_{min}$ denotes the minimum of $f(x)+f(-x)$.
\begin{proof} First let $g=f(x)+f(-x)-f_{min}$. It is easily provable that  $f(1)=f'(1)=0$ which means every positive coefficient of monomials in $f$ is minimal in that, if we would decrease them $g$ would be negative. Furthermore there now is an extra negative coefficient which means $g\not \in SAGE(A,\R)$. 
\begin{align*}
c_g=\myvec{c_3\\-c\\c_2\\c_3\\-c\\c_2\\-f_{min}}
\end{align*}
Nonnegativity follows from nonegativity of $f$.
\end{proof}
\end{thm}
A specific example of this class would be.
\begin{bsp}
Let $c=\frac 5 4 =1.25$ and $\alpha = \frac 3 2 = 1.5$ and $p(x)=ce^{\alpha x}$. By Theorem \ref{monomClass} we can construct
\begin{align*}
f(x)= \frac{5}{8e}e^{2.5x} -\frac 5 4 e^{1.5x}+\frac{5e}{8}e^{0.5x}.
\end{align*}
it holds that $f(1)=f'(1)=0$. By construction we know that $f(x)+f(-x)-f_{min}\geq 0$ for all $x \in \mathbb{R}$.
\end{bsp}
Whereas this procedure generates a huge class of examples these are not positive. Theorem \ref{SAGEthm} requires positivity for which we can construct a smaller and not globally positiv example class
\begin{thm}
Let $\mathcal{X}=[-1,1]^2$ and
\begin{align*}
f(x) = c_1e^{a_1x_1} + c_2e^{a_2x_2} + c_3 
\end{align*}
with $a_1<0, a_2>0, c_1<0, c_2<0$. If 
\begin{align*}
max \lbrace -c_1e^{-a_1} -c_2e^{-a_2},-c_1e^{-a_1} -c_2e^{a_2} \rbrace < c_3 < c_1e^{-a_1}-c_2e^{a_2}
\end{align*}
\end{thm}
\begin{proof}
ToDo...
\end{proof}



\section{Complexity}
ToDo
\end{document}